\documentclass[8pt,a4paper,landscape,oneside]{amsart}
\usepackage{amsmath, amsthm, amssymb, amsfonts}
\usepackage[T1]{fontenc}
\usepackage[utf8]{inputenc}
\usepackage{booktabs}
\usepackage{caption}
\usepackage{color}
\usepackage{fancyhdr}
\usepackage{float}
\usepackage{fullpage}
\usepackage{subcaption}
\usepackage[scaled]{beramono}
\usepackage{titling}
\usepackage{datetime}
\usepackage{enumitem}
\usepackage{multicol}
\usepackage{bm}

% Minted
\usepackage{minted}
\newcommand{\codej}[1]{\inputminted[fontsize=\large,baselinestretch=1]{java}{code/#1}}
\newcommand{\codec}[1]{\inputminted[fontsize=\large,baselinestretch=1]{cpp}{code/#1}}
\newcommand{\codep}[1]{\inputminted[fontsize=\large,baselinestretch=1]{py}{code/#1}}
\newcommand{\codeb}[1]{\inputminted[fontsize=\large,baselinestretch=1]{bash}{code/#1}}
\newcommand{\codev}[1]{\inputminted[fontsize=\large,baselinestretch=1]{vim}{code/#1}}
\newcommand{\subtitle}[1]{%
  \posttitle{%
    \par\end{center}
    \begin{center}\large#1\end{center}
    \vskip0.1em\vspace{-1em}}%
}
\pagestyle{fancy}
\lhead{Cache Flow - Lund University}
\rhead{\thepage}
\cfoot{}
\setlength{\headheight}{15.2pt}
\setlength{\droptitle}{-20pt}

\posttitle{\par\end{center}}
\renewcommand{\headrulewidth}{0.4pt}
\renewcommand{\footrulewidth}{0.4pt}

\newcommand{\bigO}{\mathcal{O}}

\title{Cache Flow}
\subtitle{Team Reference Document}
\date{\ddmmyyyydate{\today{}}}

\begin{document}

\begin{multicols*}{2}
\maketitle
\thispagestyle{fancy}
\vspace{-3em}

\tableofcontents
\begin{large}
\section{Acieving AC on a solved problem}
    \subsection{WA}
        \begin{itemize}
        \item Check that minimal input passes.
        \item Can an int overflow?
        \item Reread the problem statement.
        \item Start creating small testcases with python.
        \item Does cout print with high enough precision?
        \item Abstract the implementation.
        \end{itemize}
    \subsection{TLE}
        \begin{itemize}
        \item Is the solution sanity checked?
        \item Use pypy instead of python.
        \item Rewrite in C++ or Java.
        \item Can we apply DP anywhere?
        \item To minimize penalty time you should create a worst case input (if easy) to test on.
        \item Binary Search over the answer?
        \end{itemize}
    \subsection{RTE}
        \begin{itemize}
        \item Recursion limit in python?
        \item Arrayindex out of bounds?
        \item Division by $0$?
        \item Modifying iterator while iterating over it?
        \item Not using a well definied operator for Collections.sort?
        \item If nothing makes sense and the end of the contest is approaching you 
            can binary search over where the error is with try-except.
        \end{itemize}
    \subsection{MLE}
        \begin{itemize}
        \item Create objects outside recursive function.
        \item Rewrite recursive solution to itterative with an own stack.
        \end{itemize}
\section{Ideas}
    \subsection{A TLE solution is obvious}
        \begin{itemize}
        \item If doing dp, drop parameter and recover from others.
        \item Use a sorted datastructure.
        \item Is there a hint in the statement saying that something more is bounded?
        \end{itemize}
    \subsection{Try this on clueless problems}
        \begin{itemize}
        \item Try to interpret problem as a graph (D - NCPC2017).
        \item Can we apply maxflow, with mincost?
        \item How does it look for small examples, can we find a pattern?
        \item Binary search over solution.
        \item If problem is small, just brute force instead of solving it cleverly.
            Some times its enough to iterate over the entire domains instead of using xgcd.
        \end{itemize}
\section{Code Templates}
    \subsection{.bashrc}
        Aliases.
        \codeb{.bashrc}
    \subsection{.vimrc}
        Tabs, linenumbers, wrapping
        \codev{.vimrc}
    \subsection{Java Template}
        A Java template.
        \codej{template.java}
    \subsection{Python Template}
        A Python template
        \codep{template.py}
    \subsection{C++ Template}
        A C$++$ template
        \codec{template.cpp}

\section{Data Structures}
    \subsection{Binary Indexed Tree}
        Also called a fenwick tree. Builds in $\bigO(n \log{n})$ from an array. Querry sum from 0 to i in $\bigO(\log{n})$ and updates an element in $\bigO(\log{n})$.
        \codej{DS/BIT.java}
    \subsection{Segment Tree}
        More general than a fenwick tree. Can adapt other operations than sum, e.g.\ min and max.
        \codej{DS/ST.java}
    \subsection{Lazy Segment Tree}
        More general implementation of a segment tree where its possible to increase whole segments by some diff, with lazy propagation. Implemented with arrays instead of nodes, which probably has less overhead to write during a competition.
        \codej{DS/LazySegmentTree.java}
    \subsection{Union Find}
        This data structure is used in varoius algorithms, for example Kruskals algorithm for finding a Minimal Spanning Tree in a weighted graph. Also it can be used for backward simulation of dividing a set.
        \codej{DS/UF.java}
    \subsection{Monotone Queue}
        Used in sliding window algorithms where one would like to find the minimum in each interval of a given length. Amortized $\bigO(n)$ to find min in each of these intervals in an array of length $n$. Can easily be used to find the maximum as well.
        \codej{DS/MinMonQue.java}
    \subsection{Treap}
        Treap is a binary search tree that uses randomization to balance itself. 
        It's easy to implement, and gives you access to the internal structures of a binary tree, 
        which can be used to find the k'th element for example. Because of the randomness, the average height is about a factor 4 of a prefectly balanced tree.
        \codej{DS/Treap.java}
\section{Graph Algorithms}
    \subsection{Djikstras algorithm}
        Finds the shortest distance between two Nodes in a weighted graph in $\bigO (|E| \log{|V|})$ time.
        \codej{Graphs/Djikstra.java}
    \subsection{Bipartite Graphs}
        The Hopcroft-Karp algorithm finds the maximal matching in a bipartite graph. Also, this matching can together with Könings theorem be used to construct a minimal vertex-cover, which as we all know is the complement of a maximum independent set. Runs in $\bigO (|E|\sqrt{|V|})$. 
        \codej{Graphs/BiGraph.java}
    \subsection{Network Flow}
        The Floyd Warshall algorithm for determining the maximum flow through a graph can be used for a lot of unexpected problems. Given a problem that can be formulated as a graph, where no ideas are found trying, it might help trying to apply network flow. The running time is $\bigO (C \cdot m)$ where $C$ is the maximum flow and $m$ is the amount of edges in the graph. 
        If $C$ is very large we can change the running time to $\bigO (\log{C}m^2)$ by only studying edges with a large enough capacity in the beginning.
        \codej{Graphs/NetworkFlow.java}
    \subsection{Min Cost Max Flow}
        Finds the minimal cost of a maximum flow through a graph. 
        Can be used for some optimization problems where the optimal assignment needs to be a maximum flow.
        \codej{Graphs/MinCostMaxFlow.java}

\section{Dynamic Programming}
    \subsection{Longest Increasing Subsequence}
        Finds the longest increasing subsequence in an array in $\bigO(n \log{n})$ time. Can easility be transformed to longest decreasing/nondecreasing/nonincreasing subsequence.
        \codej{DP/lis.java}
    \subsection{String functions}
        The z-function computes the longest common prefix of $t$ and $t[i:]$ for each $i$ in $\bigO(|t|)$.
        The border function computes the longest common proper (smaller than whole string) prefix and suffix of string $t[:i]$.
        \codep{DP/strings.py}
    \subsection{Josephus problem}
        Who is the last one to get removed from a circle if the k'th element is continously removed?
        \codep{DP/josephus.py}

\section{Etc}
    \subsection{System of Equations}
        Solves the system of equations $\bm{A}\bm{x} = \bm{b}$ by Gaussian elimination. This can for example be used to determine the expected value of each node in a markov chain. Runns in $\bigO (N^3)$.
        \codej{Etc/SolveSystem.java}
    \subsection{Convex Hull}
        From a collection of points in the plane the convex hull is often used to compute the largest distance or the area covered, or the length of a rope that encloses the points. It can be found in $\bigO (N\log{N})$ time by sorting the points on angle and the sweeping over all of them.
        \codej{Etc/ConvexHull.java}
    \subsection{Number Theory}
        \codep{Etc/numbertheory.py}

\section{NP tricks}
    \subsection{MaxClique}
        The max clique problem is one of Karp's 21 NP-complete problems. The problem is to find the lagest subset of an undirected graph that forms a clique - a complete graph. There is an obvious algorithm that just inspects every subset of the graph and determines if this subset is a clique. This algorithm runns in $\bigO(n^22^n)$. However one can use the meet in the middle trick (one step divide and conqurer) and reduce the complexity to $\bigO(n^22^{\frac{n}{2}})$.
        \codej{NP/MaxClique.java}
\section{Coordinate Geometry}
    \subsection{Area of a nonintersecting polygon}
        The signed area of a polygon with n verticies is given by 
        $$A = \frac{1}{2}\sum_{i=0}^{n-1}(x_iy_{i+1} - x_{i+1}y_i)$$
    \subsection{Intersection of two lines}
        Two lines defined by 
        \begin{align*}
            a_1x + b_1y + c_1 &= 0 \\
            a_2x + b_2y + c_2 &= 0 
        \end{align*}
        Intersects in the point 
        $$P = (\frac{b_1c_2 - b_2c_1}{w}, \frac{a_2c_1 - a_1c_2}{w}),$$
        where $w = a_1b_2 - a_2b_1$. If $w = 0$ the lines are parallell.
        \subsection{Distance between line segment and point}
        Given a linesegment between point $P, Q$, the distance D to point $R$ is given by:
        \begin{align*}
            a &= Q_y - P_y \\
            b &= Q_x - P_x \\
            c &= P_xQ_y - P_yQ_x \\
            R_P &= (\frac{b(bR_x - aR_y) - ac}{a^2 + b^2}, \frac{a(aR_y - bR_x) - bc}{a^2 + b^2}) \\
            D &= 
            \begin{cases}
                \frac{|aR_x + bR_y + c|}{\sqrt{a^2 + b^2}} & \text{if $(R_{P_x}- P_x)(R_{P_x} - Q_x) < 0$}, \\
                \min{|P - R|, |Q - R|} & \text{otherwise}
            \end{cases}
        \end{align*}
    \subsection{Picks theorem}
        Find the amount of internal integer coordinates $i$ inside a polygon with picks theorem
        $A = \frac{b}{2} + i - 1$, where $A$ is the area of the polygon and 
        $b$ is the amount of coordinates on the boundary.
    \subsection{Implementations}
        \codep{Geometry/geometry.py}
\newpage
\section{Practice Contest Checklist}
\begin{itemize}
    \item Operations per second in py2
    \item Operations per second in py3
    \item Operations per second in java
    \item Operations per second in c++
    \item Operations per second on local machine
    \item Is MLE called MLE or RTE?
    \item What happens if extra output is added? What about one extra new line or space?
    \item Look at documentation on judge.
    \item Submit a clar.
    \item Print a file.
    \item Directory with test cases.
\end{itemize}


\end{large}
\end{multicols*}
\end{document}
